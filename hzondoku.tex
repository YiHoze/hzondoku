\documentclass[b5paper]{article}
\usepackage{hzondoku}

\setmainfont{NotoSerifCJKjp-Regular.otf}[BoldFont=NotoSerifCJKjp-Bold.otf]
\setsansfont{NotoSansCJKjp-Regular.otf}[BoldFont=NotoSansCJKjp-Bold.otf]

\setlength\parindent{0pt}
\renewcommand\baselinestretch{2}

\title{hzondoku 패키지}
\author{이 호재}
\date{}

\begin{document}

\maketitle

hzondoku 패키지는 히라가나와 카타카나의 음을 로마자로 표시하는 \verb|\ondoku| 명령을 제공한다.
이 패키지는 또한 \verb|\ruby| 명령을 재정의하여 토의 음을 표시한다.
다른 글꼴이나 색으로 음이 표시되게 하려면 \verb|\textod| 명령을 재정의하라.
\verb|\HideOndoku| 명령이 선언되면 \verb|\ondoku| 명령이 무효해져 음이 표시되지 않는다.

\begin{verbatim}
\ondoku{ひらがな}
\ondoku*{ひらがな}
\ruby{拗音}{ようおん}
\ruby*{拗音}{ようおん}
\end{verbatim}

\ondoku{ひらがな}
\ondoku*{ひらがな}
\ruby{拗音}{ようおん}
\ruby*{拗音}{ようおん}

\twocolumn

\section{拗音}

\ruby{拗音}{ようおん}은 い 단 뒤에 오는 や·ゆ·よ이다.

\ruby{茶}{ちゃ} 
\ruby{写真}{しゃしん} 

\section{促音}

\ruby{促音}{そくおん} 뒤에 오는 글자에 따라 그 자음이 앞 자에 추가되어 받침소리처럼 들린다.

\ruby{殺気}{さっき} 
\ruby{先}{さっき} 

\ondoku{いっそ} 차라리, 실로  
\ondoku{ずっと} 훨씬, 쭉 
\ruby{立派}{りっぱ} 훌륭함 

\section{長音}

\ruby{長音}{ちょうおん}

앞 자와 뒤 자가 같은 모음으로 끝나면 장음으로 바뀐다.

\ondoku{さあ} \ondoku{とお}

え 단 글자 뒤에 오는 い는 장음으로 바뀐다.

\ruby{先生}{せんせい}

お 단 글자 뒤에 오는 う는 장음으로 바뀐다.

 こうどう

\section{発音}

ん \ruby{発音}{はつおん}은 뒤 자에 따라 n, ng, ñ, m으로 바뀐다.
ñ은, 솔직히 어떻게 발음하는지 모르겠지만, `응'과 `은'의 중간 소리를 나타낸다.

\ondoku{でんわ} 
\ondoku{にほん} 

\ondoku{おんな} 
\ondoku{きんじょ} 
\ondoku{こんだん} 

\ondoku{ぎんこう} 
\ondoku{りんご} 

\ondoku{じゅんび} 
\ondoku{うんめい}

\newpage

\section{ひらがな} 

\ondoku{あ い う え お}\\
\ondoku{か き く け こ}\\
\ondoku{が ぎ ぐ げ ご}\\
\ondoku{さ し す せ そ}\\
\ondoku{ざ じ ず ぜ ぞ}\\
\ondoku{た ち つ て と}\\
\ondoku{だ ぢ づ で ど}\\
\ondoku{な に ぬ ね の}\\
\ondoku{は ひ ふ へ ほ}\\
\ondoku{ば び ぶ べ ぼ}\\
\ondoku{ぱ ぴ ぷ ぺ ぽ}\\
\ondoku{ま み む め も}\\
\ondoku{ら り る れ ろ}\\
\ondoku{わ を} \\
\ondoku{ん } \\ 
\ondoku{や ゆ よ } 

\newpage

\section{かたかな}

\ondoku{ア イ ウ エ オ}\\
\ondoku{カ キ ク ケ コ}\\
\ondoku{ガ ギ グ ゲ ゴ}\\
\ondoku{サ シ ス セ ソ}\\
\ondoku{ザ ジ ズ ゼ ゾ}\\
\ondoku{タ チ ツ テ ト}\\
\ondoku{ダ ヂ ヅ デ ド}\\
\ondoku{ナ ニ ヌ ネ ノ}\\
\ondoku{ハ ヒ フ ヘ ホ}\\
\ondoku{バ ビ ブ ベ ボ}\\
\ondoku{パ ピ プ ペ ポ}\\
\ondoku{マ ミ ム メ モ}\\
\ondoku{ラ リ ル レ ロ}\\
\ondoku{ワ ヲ} \\
\ondoku{ン } \\ 
\ondoku{ヤ ユ ヨ } 

\end{document}
